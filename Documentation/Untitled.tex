\documentclass{article}
\usepackage[margin=0.5in]{geometry}
\usepackage{float}
\usepackage{hyperref}
\title{Lake Shore Temperature Ramp\\ Documentation}
\date{\today}
\author{J. Sheldon}
\begin{document}
\maketitle
\section{Brief Overview}
This Python script serves as automated way to step through, read several channels' resistances / temperatures, and plot data from the Lake Shore Model 372.  The end user will need to have several dependencies installed on their system, edit the `config.ini' file provided with the Python code to suit their needs, and run the python code. The resulting output is a `live-ish' plot and a comma separated list in `.csv' format (naming convention `YYYYMMDD-HHmm\_UserDescription.csv') located in a folder `captures'. This document serves as instructions for typical use and for future reference in development.
\section{Using The Script}
\subsection{Install Directions}
This software should be OS agnostic, however you will need to install several pieces of software and python packages for it to work. It is suggested to install `Anaconda, Python 2.7 version' to simplify the install process. You can find that here: \url{https://www.continuum.io/downloads}. Following that, you will need to install `numpy', `pandas',  and `configparser'. You can install these in Anaconda Navigator's `Environment' menu. You will also want to add Anaconda to either your user PATH or system PATH variable. Check how to do this online. After completing these steps, your software environment should be set up to run the application \\

\subsection{Usage}
You are almost ready! The next to final step is to edit the `config.ini' file to your specific needs. When you are ready to use the application, please run `python ls\_tempramp.py' in your system shell (CMD or suitable terminal). The script will ask you for an identifier/description -- this is meant to be a short description of what you are testing. This will be the `UserDescription' part of the filename. Your script will run and display a progressbar in the shell and should shortly plot some data. After completion, note that you should now have a plot and a `.csv'  file in the `captures' folder. (You can zoom/pan around in the plot. You may wish to save the plot immediately -- unlike the `.csv' file, it is not saved automatically.)

\subsection{Configuration File}
The `config.ini' file holds all of the configuration data for your run. These are the units and descriptions for each variable stored in the file: 

\begin{figure}[H]
\centering\begin{tabular}{ | r | c | l  |}
Section & Variable & Description \\ \hline \hline
connection & serialport & Port to connect to (eg. `COM3' or `/dev/ttyS3') \\
	           & baudrate & communication rate (eg. 9600 or 115200)\\ \hline
sampleheater & resistance & Resistance of heater (in Ohms)\\
		      & maxcurrent & current limit (in Amps)\\
		      & range & integer 0-8. See \autoref{fig:range}\\
		      & deltapc & percentage of current range applied to heater (0.0-100.0)\\
		      & initpc & initial current percentage (0.0-100.0)\\
		      & finalpc & final current percentage (0.0-100.0) \\ 
		      & disp & 0 for current, 1 for power \\ \hline
scanner	      & autoscan & 1 for autoscanning, 0 for no autoscanning \\ \hline
mcthermometer & channel & mixing chamber thermometer channel\\
		      & tempcoeff & mixing chamber thermometer temperature coefficient (0=neg, 1=pos)\\
		      & curvenumber & mixing chamber thermometer calibration curve number \\
		      & filterwindow & windowing percentage of filter (0.0-100.0 allowed) \\ 
		      & t\_settle & Filter settle time\\
			& t\_dwell & Scanner dwell time\\
			& t\_pause & Scanner pause time \\ \hline
sample1	      & channel & Scanner channel\\
		      & tempcoeff & temperature coefficient\\	    
		      & curvenumber & calibration curve number \\  
		      & filterwindow & windowing percentage of filter (0.0-100.0 allowed) \\
		      & description & Short descriptor \\
		      & t\_settle & Filter settle time\\
			& t\_dwell & Scanner dwell time\\
			& t\_pause & Scanner pause time \\ \hline
sample2	      & channel & Scanner channel\\
		      & tempcoeff & temperature coefficient\\	    
		      & curvenumber & calibration curve number \\  
		      & filterwindow & windowing percentage of filter (0.0-100.0 allowed) \\
		      & description & Short descriptor \\
		      & t\_settle & Filter settle time\\
			& t\_dwell & Scanner dwell time\\
			& t\_pause & Scanner pause time \\ \hline
timeconstants & t\_therm & Thermalization time\\
			& t\_switch & Switching time\\ \hline
\end{tabular}
\caption{Variables stored in the `config.ini' file}
\end{figure}

\section{Development Notes}
\subsection{Class: LakeShore372}
\subsubsection{Description / Use}
This object holds member functions for opening/closing of a serial port in addition to a simple command set used to communicate with the Lake Shore AC / Resistance Bridge.

\subsubsection{Member Variables}
\begin{figure}[H]
\centering
\begin{tabular}{|c | c | c | }
Variable & Datatype & Description \\ \hline \hline
ID & string & ID of the instrument \\
sampleheater & dictionary & sample heater prefs. \\
scanner & dictionary & scanner prefs \\
mcthermo & dictionary & mixing chamber thermometer prefs\\
sample1 & dictionary & sample 1 prefs\\
sample2 & dictionary &  sample 2 prefs \\
timeConstants & dictionary & global time constants\\ \hline

\end{tabular}
\caption{Built in variables / objects for the LakeShore372 class. The dictionary indices share the names of the variables in the ini file.}
\end{figure}

\subsubsection{Member Functions}
\begin{figure}[H]
\centering
\begin{tabular}{| r | l | }
Member Function & Description \\ \hline \hline
open(x,y)  & Open a serial connection on port `x' with baudrate `y'\\
close()  & Close the serial connection\\
getConfig(x) & Loads in the ini file `x'. \\
setCHParams(x,y,z,a,b,c) & Sets CH `x' to have a dwell time `y', pause time `z', curve number `a', \\
 & temp coeff `b' (0=neg,1=pos). If `c' = 1, channel is on, 0 it is off.\\
setFilterParams(w,x,y,z) & sets ch `w' filter to have settle time `x', window `y'. Filter on if `z' = 1, filter off if `z' = 0\\
ReadCHStatus(x)  & Read the status of the channel. See below for how to interpret response\\
ReadResistance(x)  & Reads the resistance of ch `x'\\
ReadKelvin(x)  & Reads the temperature of ch `x'\\
ScanTo(x,y)  & Scans to ch `x'. Autoscan is on if $y = 1$, off if $y=0$\\
SetSampleHeaterRange(x)  & Sets heater current range. See below for valid inputs. \\
SetSampleHeaterCurrent(x,y,z)  & Sets  sample heater (Resistance = `x')  maximum current to percentage\\
&  (inputs 0-100 valid) of range. $z=1$ sets display mode to current, $0$ to power \\ \hline
\end{tabular}
\caption{Built in member functions for the LakeShore372 class}
\end{figure}

\begin{figure}[H]
\centering
\begin{tabular}{r | l }
Return & Meaning \\ \hline \hline
1 & CS OVL\\
2 & VCM OVL\\
4 & VMIX OVL\\
8 & VDIF OVL\\
16 & R. OVER\\
32 & R. UNDER\\
64 & T. OVER\\
128 & T. UNDER\\
\end{tabular}
\caption{Return values for ReadCHStatus(x) and their meanings}
\end{figure}

\begin{figure}[H]
\centering
\begin{tabular}{r | l }
Integer Input & Desired Range \\ \hline \hline
0 & off \\
1 & 31.6 $\mu A$ \\
2 & 100 $\mu A$\\
3 & 316 $\mu A$\\
4 & 1.0 $mA$ \\
5 & 3.16 $mA$\\
6 & 10.0 $mA$\\
7 & 31.6 $mA$ \\
8 & 100 $mA$\\
\end{tabular}
\label{fig:range}
\caption{`x' inputs for SetSampleHeaterRange(x) and their corresponding ranges.}
\end{figure}


\subsection{Class: LakeShore372Data}
\subsubsection{Description / Use}
This object largely serves as a data handler for the data coming from the Lake Shore device, and contains several functions for manipulating / saving this data.
\subsubsection{Member Variables}
\begin{figure}[H]
\centering\begin{tabular}{c | c | l }
Member Vars / Objects & Datatype & Description \\ \hline \hline
MCThermoR[ ]& float & list of resistances for the mixing chamber thermometer \\
MCThermoK[ ]& float & list of temperatures for the mixing chamber thermometer \\
Sample1R[ ] & float & list of resistances for sample 1 \\
Sample2R[ ] & float & list of resistances for sample 2 \\
MCThermoRL& float & last read resistances for the mixing chamber thermometer \\
MCThermoKL& float & last read temperatures for the mixing chamber thermometer \\
Sample1RL& float & last read resistance for sample 1 \\
Sample2RL & float & last read resistance for sample 2 \\
DataFile & file stream & CSV file to write to
\end{tabular}
\caption{Member variables/objects for the LakeShore372Data class.}
\end{figure}

\subsubsection{Member Functions}
\begin{figure}[H]
\centering\begin{tabular}{c | c | c }
Member Function & Description \\ \hline \hline
 & string & ID of the instrument \\
\end{tabular}
\caption{Member functions for the LakeShore372Data class}
\end{figure}



\end{document}